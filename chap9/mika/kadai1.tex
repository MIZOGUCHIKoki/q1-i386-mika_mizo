\chapter{コンパイラを使用せず,アセンブリ,機械語のみでのソートアルゴリズムの記述}
\vspace{-1cm}
\section{目的}
この実験の目的は,「コンパイラを使用せず,アセンブリ,機械語のみでソートアルゴリズムを記述することが可能である」ことを示すことである.

\section{方法}
第1章に示した条件を満たすよう,高級言語で記述したものと同じ挙動をするようなソートアルゴリズムを記述する.また,使用した実行コマンドは以下である.
print\_eax.s,test\_sort.s,sort.sに対して以下を行う.
\begin{lstlisting}[numbers={none}, caption={実行コマンド}, label={fig:実行コマンド}]
$ nasm -felf print_eax.s
$ nasm -felf test_sort.s
$ nasm -felf sort.s
$ ld -m elf_i386 print_eax.o test_sort.o sort.o
$ ./a.out
\end{lstlisting}

\section{実験結果}
\begin{table}[h]
  \centering
   \begin{minipage}{0.45\textwidth}
     a.outを実行した結果,右のような結果が得られた.\\
     この結果から,正しく動作していると考えられる.
   \end{minipage}
   \hfill
   \begin{minipage}{0.45\textwidth}
     \centering
     \begin{tabular}{c|cccccccccccc}
       入力 & 1 & 3 & 5 & 7 & 9 & 2 & 4 & 6 & 8 & 0 & 1 & 2\\\hline
       出力 & 0 & 1 & 1 & 2 & 2 & 3 & 4 & 5 & 6 & 7 & 8 & 9\\
       期待出力 & 0 & 1 & 1 & 2 & 2 & 3 & 4 & 5 & 6 & 7 & 8 & 9\\
     \end{tabular}
   \end{minipage}
\end{table}

\section{考察}
実験結果から,「1以上30万個以下の0以上$2^{32}$未満のダブルワード」に関してはアセンブリ言語でソートアルゴリズムが記述可能であると言える.この条件を満たさない場合のソートアルゴリズムも記述可能であるかは定かでない.
