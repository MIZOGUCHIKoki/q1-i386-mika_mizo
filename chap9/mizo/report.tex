\RequirePackage{luatex85}
\documentclass[paper=a4,11pt]{jlreq}
\usepackage{amsmath,amsfonts,amssymb,mathtools,ascmac,bm,fancybox,calc,multicol,array}
\usepackage[top=20truemm,bottom=20truemm,left=15truemm,right=15truemm]{geometry}
\usepackage{graphicx,color}
\usepackage{tikz,listings,wrapfig,float,xcolor}
\usepackage{url,subcaption,multirow,framed}
\usepackage[unicode=true,hidelinks,pdfusetitle]{hyperref}
\usepackage{luatexja-fontspec}
\usepackage{tcolorbox,listings}
\hypersetup{
    colorlinks=true,
    citecolor=black,
    linkcolor=black,
    urlcolor=blue
}
    \usetikzlibrary{intersections,calc,arrows.meta,backgrounds,shapes.geometric,shapes.misc,positioning,fit,graphs,arrows}
    \tcbuselibrary{raster,skins,breakable,theorems}
    \setlength{\columnsep}{5mm}
    \columnseprule=0.1mm
    \renewcommand{\indent}{1\zw}
    \setlength{\parindent}{1\zw}
    \ltjsetparameter{jacharrange={-2}} %日本語以外を欧文扱い

    \renewcommand{\thefootnote}{*\arabic{footnote}}
\renewcommand{\lstlistingname}{}
\renewcommand{\figurename}{}
\renewcommand{\tablename}{}
\AtBeginDocument{
            \renewcommand{\thelstlisting}{src \thesection.\arabic{lstlisting}}
}
\lstset{
        %プログラム言語(複数の言語に対応,C,C++も可)
    language = {[x86masm]Assembler},
        %背景色と透過度
    %backgroundcolor={\color[gray]{.90}},
        %枠外に行った時の自動改行
    breaklines = true,
        %自動改行後のインデント量(デフォルトでは20[pt])
    breakindent = 10pt,
        %標準の書体
    basicstyle = \ttfamily\small,
        %コメントの書体
    commentstyle = {\ttfamily \color[cmyk]{1,0.4,1,0}},
        %関数名等の色の設定
    classoffset = 0,
        %キーワード(int, ifなど)の書体
    keywordstyle = {\bfseries \color[cmyk]{0,1,0,0}},
        %表示する文字の書体
    stringstyle = {\ttfamily \color[rgb]{0,0,1}},
        %枠 tは上に線を記載, Tは上に二重線を記載
        %他オプション:leftline,topline,bottomline,lines,single,shadowbox
    frame = lines,
        %frameまでの間隔(行番号とプログラムの間)
    framesep = 5pt,
        %行番号の位置
    numbers = left,
        %行番号の間隔
    stepnumber = 1,
        %行番号の書体
    numberstyle = \small,
        %タブの大きさ
    tabsize = 4,
        %キャプションの場所(tbならば上下両方に記載)
    captionpos = t
}
\lstdefinelanguage{Bash}{
    morekeywords=[1]{
        nasm,ld,uname
    },
    sensitive=true
}

\makeatletter
    \renewcommand{\thefigure}{%
    Fig\ \thesection.\arabic{figure}}
    \@addtoreset{figure}{section}
    \renewcommand{\thetable}{%
    Tbl\ \thesection.\arabic{table}}
    \@addtoreset{table}{section}

    \@addtoreset{lstlisting}{section}
\makeatother
\title{\vspace{-2cm}{\normalsize 情報学群実験第2 レポート}\\\vspace{0.5em}アセンブリ言語による整列アルゴリズムに関する実験}
\author{1250373 溝口洸熙\thanks{高知工科大学 情報学群 2年 清水研究室}}
\date{\today}

\begin{document}
\maketitle
\pagenumbering{roman}\pagestyle{plain}
\setcounter{tocdepth}{3}
\tableofcontents

\newpage
\newcommand{\testsort}{{\ttfamily test\_sort.s}}
\newcommand{\sort}{{\ttfamily sort}}
\newcommand{\print}{{\ttfamily print\_eax}}
\setcounter{page}{0}
\pagenumbering{arabic}
\chapter{アセンブリ言語による整列アルゴリズム記述可否の検証}\label{chap1}
\section{実験の目的}
高級プログラミング言語,{\ttfamily Java, C, Python}などは,『コンパイラ』と呼ばれる装置を通して機械語に書き換えられ,コンピュータで実行されている.\par
それに対して,アセンブリ言語は各機械語命令につけられた「意味する名前」(ニーモニック;mnemonic)を使ってプログラムを表記する表記法である.\cite[第1章]{pl2text}
また,アセンブリ言語表記を機械語のビット列に変換する作業をアセンブルと言い,それを行うソフトウェアをアセンブラと言う.
つまり,コンパイラとアセンブラは別物であり,アセンブラは機械語の表記を変えたものである故にコンピュータへの命令を1対1で書き換えるものである点がコンパイラと大きく違う点である.\par
本実験課題の目的は,このようなアセンブリ言語・機械語に対して,コンパイラを使わずに整列アルゴリズムを直接技術することが可能であるか確認することである.
\section{プログラムの仕様}
\(0\)以上\(2^{31}\)未満の自然数列に対して昇順に整列するアルゴリズムをアセンブリ言語で記述する.その際,\testsort ファイルが\sort サブルーチンを呼び出して整列を行う.
検証したい自然数列は,\testsort 内の{\ttfamily data1}にダブルワードで,データの個数は{\ttfamily ndata1}で定義しており,\testsort 実行時{\ttfamily data1}に格納してある自然数列が\print サブルーチンによって出力される.\par
{\ttfamily data1}の先頭番地は{\ttfamily EBX},{\ttfamily ndata1}は{\ttfamily ECX}に格納する.\sort の呼び出し前後で他の汎用レジスタの値は変化しないように設計されている.内部処理の概要を\ref{kadai1:abs}に示す.整列アルゴリズムは,\textbf{選択ソートアルゴリズム}\footnote{入力データの最大値を見つけ,それを整列アルゴリズムから除外することを繰り返し行うアルゴリズム\cite[p.49]{アルゴリズムとデータ構造}}を元に作成している.
\begin{figure}[H]
    \centering
    \caption{処理概要}
    \label{kadai1:abs}
    \tikzset{mynode/.style={rectangle,rounded corners,draw,minimum height=1cm,minimum width=3cm,text centered}}
    \begin{tikzpicture}
        \node at (0,0)(call){{\ttfamily call sort}};
        \node[below=0.5cm of call](print){{\ttfamily call print\_eax}};
        \node[right=2cm of call,text width=5cm](sort){{\ttfamily sort}サブルーチン};
        \node[left=2cm of print,text width=5cm](printeax){\print サブルーチン};
        \draw[-Stealth,very thick](call.east)to(sort.west);
        \draw[-latex,very thick](call)to(print);
        \draw[-Stealth,very thick](print)to[bend left=30](printeax);
        \draw[-Stealth,very thick](printeax)to[bend left=30](print);
        \node at ($(print)!0.5!(printeax)+(0,-1.5cm)$){\sort されたデータを1行ずつ表示};
        \node[inner sep=0.2cm,fit={(print)(call)},draw,dotted](){};
    \end{tikzpicture}
\end{figure}
\section{実験}
\subsection{実験方法}
\ref{usingPC}に示すコンピュータとソフトウェアを利用して,アセンブリ言語で書いた整列プログラムを実行可能ファイルに変換し実行した.
実行結果を確認し,整列されているか確認するとともに,{\ttfamily Java}で記述したプログラムとアセンブリプログラムを比較してどの部分をどのようにアセンブリ言語化したかを確認する.
\begin{lstlisting}[frame={single},numbers={none},breakindent={0pt},language={Bash},caption={使用したコンピュータとソフトウェア},label={usingPC}]
$ uname -a
Linux KUT20VLIN-462 5.4.0-70-generic #78~18.04.1-Ubuntu SMP Sat Mar 20 14:10:07 UTC 2021 x86_64 x86_64 x86_64 GNU/Linux
$ nasm --version
NASM version 2.13.02
$ ld --version
GNU ld (GNU Binutils for Ubuntu) 2.30
$ java --version
openjdk 11.0.10 2021-01-19
\end{lstlisting}
\print{\ttfamily .s},\testsort,\sort{\ttfamily .s}をそれぞれアセンブルして実行する.
\begin{lstlisting}[frame={single},numbers={none},breakindent={0pt},language={Bash},caption={実行したコマンド},label={command1}]
$ nasm -felf print_eax.s ; nasm -felf test_sort.s ; nasm -felf sort.s
$ ld -m elf_i386 print_eax.o test_sort.o sort.o
$ ./a.out
\end{lstlisting}
\subsection{実験結果}
実験結果と期待される結果を\ref{tbl:execute}に示す.自然数列を昇順に整列されていることが確認できる.\par
\begin{table}[H]
    \centering
    \caption{実験結果と比較}
    \label{tbl:execute}
    \begin{tabular}{l|l}
        入力   & {\ttfamily 1 3 5 7 9 2 4 6 8 0 1 2} \\
        \hline
        出力   & {\ttfamily 0 1 1 2 2 3 4 5 6 7 8 9} \\
        期待出力 & {\ttfamily 0 1 1 2 2 3 4 5 6 7 8 9}
    \end{tabular}
\end{table}
{\ttfamily Java}で記述した選択ソート(\ref{src:java_buble})と,アセンブリ言語で記述した選択ソート(\ref{src:assembuly_buble})を比較して,言語化の箇所を確認する.前者は入力データ列を{\ttfamily data}とし,後者は{\ttfamily EBX}に入力データ列の先頭番地,{\ttfamily ECX}入力データ列の個数を格納している.ソースコードの各処理の内容と,2つの言語による対応を\ref{tbl:ソースコードの行対応}に記す.
\begin{table}[htbp]
    \centering
    \caption{ソースコードの行対応}
    \label{tbl:ソースコードの行対応}
    \begin{tabular}{p{10cm}p{2cm}p{2cm}}
        \multicolumn{1}{c}{処理内容}                              & \multicolumn{1}{c}{\ref{src:java_buble}} & \multicolumn{1}{c}{\ref{src:assembuly_buble}} \\
        \hline
        ループ処理1の条件比較・変数処理                                      & 3                                        & 2 - 3, 25, 26                                 \\
        最大値の更新                                                & 4                                        & 5                                             \\
        最大値インデックスの更新                                          & 5                                        & 6                                             \\
        ループ処理2の条件比較・変数処理                                      & 6                                        & 7 - 9, 18, 19                                 \\
        最大値の比較・更新                                             & 7 - 11                                   & 10 - 17                                       \\
        {\ttfamily data[max\_index]}と{\ttfamily data[i]}の入れ替え & 12 - 14                                  & 21 - 24                                       \\
        \hline
    \end{tabular}
\end{table}\\
\begin{minipage}[t]{0.5\linewidth}
    \centering
    \begin{lstlisting}[caption={{\ttfamily Java}}, label={src:java_buble}, language={Java},frame={left}]
int max_index = 0;
int max = 0;
for (int i = data.length - 1; i > 0; i--) {
  max = data[0];
  max_index = 0;
  for (int j = 1; j <= i; j++) {
      if (data[j] >= max) {
              max = data[j];
              max_index = j;
          }
      }
      int m = data[max_index];
      data[max_index] = data[i];
      data[i] = m;
}
    \end{lstlisting}
    \begin{flushleft}
        \begin{framed}
            \noindent\textbf{注)}\par
            \ref{src:java_buble},\ref{src:assembuly_buble}いずれも,整列アルゴリズムの部分のみ掲載している.
        \end{framed}
    \end{flushleft}
\end{minipage}
\hspace{3em}
\begin{minipage}[t]{0.4\linewidth}
    \begin{lstlisting}[frame={left},caption={アセンブリ},label={src:assembuly_buble}]
loop0:
  cmp   ecx,  0
  jle   endp
  mov   edx,  [ebx]
  mov   eax,  0    
  mov   edi,  1
  loop1:
    cmp   edi,  ecx
    jg    loop0l
    mov   esi,  [ebx + edi*4]
    cmp   esi,  edx
    jge   then
    jmp   endif
    then:
      mov edx,  [ebx + edi*4]
      mov eax,  edi          
    endif:
      inc edi
      jmp loop1
  loop0l:
    mov   esi,  [ebx + eax*4]
    mov   edi,  [ebx + ecx*4]
    mov   [ebx + eax*4],  edi
    mov   [ebx + ecx*4],  esi
    dec   ecx
    jmp   loop0
\end{lstlisting}
\end{minipage}
\section{考察}
実験結果より自然数列を整列アルゴリズムをアセンブリ言語で記述できることがわかった.\par
ただ,これはあくまで\(0\)以上\(2^{31}\)未満の自然数に限った整列アルゴリズムであるため,負の整数やその他の有理数などを対象にした整列アルゴリズムが記述可能であるかは,この実験では検証できていない.
\newpage
\section*{謝辞}
\addcontentsline{toc}{section}{謝辞}
本実験課題は本学情報学群 1250372 三上柊氏と共同で実施した.また本学情報学群の高田 喜朗准教授には,整列アルゴリズムに関する様々な助言をいただいた.
これらの方々に深く感謝いたします.
\newpage
\renewcommand{\thesection}{\Alph{section}}
\chapter*{付録}
\addcontentsline{toc}{chapter}{付録}
\setcounter{section}{0}
\section{ソースコード}
\renewcommand{\thelstlisting}{\thesection.\arabic{lstlisting}}
\begin{lstlisting}[caption={{\ttfamily sort.s}},label={src:sort.s},frame={shadowbox}]
    section .text
    global  sort
  sort:
    push  esi
    push  edi
    push  edx
    push  ecx
    push  ebx
    push  eax
    dec   ecx
  loop0:
    cmp   ecx,  0 ; ecx = i
    jle   endp
    mov   edx,  [ebx]   ; max = data[0]
    mov   eax,  0       ; max_indent = 0
    mov   edi,  1 ; edi = j
    loop1:
      cmp   edi,  ecx   ; j > i?
      jg    loop0l
      mov   esi,  [ebx + edi*4] ; data[j]
      cmp   esi,  edx       ; data[j] >= max
      jge   then
      jmp   endif
      then:
        mov edx,  [ebx + edi*4] ; max = data[j]
        mov eax,  edi           ; max_index = j
      endif:
        inc edi
        jmp loop1
    loop0l:
      mov   esi,  [ebx + eax*4]   ; m = data[max_index]
      mov   edi,  [ebx + ecx*4]   ; edi = data[i]
      mov   [ebx + eax*4],  edi   ; data[max_index],  data[i]
      mov   [ebx + ecx*4],  esi   ; data[i] = m
      dec   ecx
      jmp   loop0
  endp:
    pop eax
    pop ebx
    pop ecx
    pop edx
    pop edi
    pop esi
    ret
\end{lstlisting}
\begin{lstlisting}[caption={{\ttfamily test\_sort.java}},label={src:testsort.s},frame={shadowbox}]
  section .text
  global  _start
  extern  sort, print_eax

_start:
  mov ebx,  data
  mov ecx,  ndata
  call  sort      ; ソート
  mov edi,  0      
loop:   ; 結果の出力
  cmp edi,  ndata
  je  endp
  mov eax,  [data + edi * 4]
  call print_eax
  inc edi
  jmp loop


endp:
  mov eax,  1
  mov ebx,  0
  int 0x80

  section .data
data:  dd 1,3, 5, 7, 9, 2, 4, 6, 8, 0, 1, 2
ndata  equ ($ - data) / 4  
\end{lstlisting}
\begin{lstlisting}[language={Java},caption={{\ttfamily sort.java}},label={src:sort.java},frame={shadowbox}]
class sort{
    public static void main(String[] args){
        int[] data = {1, 3, 5, 7, 9, 2, 4, 6, 8, 0, 1, 2};
        buble(data);
        for(int d: data){
            System.out.println(d);// 出力
        }
    }
    private static void buble(int[] data) {
        int max_index = 0; int max = 0;
        for (int i = data.length - 1; i > 0; i--) {
            max = data[0]; max_index = -0;
            for (int j = 1; j <= i; j++) {
                if (data[j] >= max) { max = data[j]; max_index = j; }
            }
            int m = data[max_index]; data[max_index] = data[i]; data[i] = m;
        }
    }
}
\end{lstlisting}
\end{document}