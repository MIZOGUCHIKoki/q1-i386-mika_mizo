\chapter{低級言語での理論的計算量に関する実験}
\section{実験の目的}
今回の実験の目的は,アセンブリ言語・機械語で直接記述した場合も実行時間は論理的計算量(アルゴリズムによって\(O(n^2)\)や\(O(n\log n)\)など)に従うことを示すことである.
\section{実験方法}
実験環境は,\ref{usingPC}に示した通り.アセンブリ言語での実行時間の検証は,Linux標準の{\ttfamily time}コマンドを用いて計測する.\ref{command1}のコマンドを実行し,実行ファイル{\ttfamily a.out}を生成した後,\ref{src:command3}を実行し実行時間を計測する.
各実行時間の中でも{\ttfamily real}が引数コマンドを実行するのにかかった時間である.実験回数は1つのテストにつき3回行い実行時間を平均する.
\begin{lstlisting}[language={Bash},caption={実行コマンド},label={src:command3},frame={single},numbers={none}]
$ time ./a.out
-- 実行結果(略)--
real    0m0.002s
user    0m0.001s
sys     0m0.000s
\end{lstlisting}
テストするデータ数は以下のように定める.
\begin{align*}
    T_1       & =\{10^n\mid n\in\mathbb{N}, n\leq 5\} \\
    T_2       & =\{n\times 10^4\mid 2\leq n\leq 9\}   \\
    {テストデータ数} & =T_1\cup T_2
\end{align*}
