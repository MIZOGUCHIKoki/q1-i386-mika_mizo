\section{アセンブリ言語と高級言語の実装に関する違い}
\subsection{実験の目的}
先にも述べたように,アセンブリ言語は機械語を1対1に記した記法である.それに対して高級言語(高水準言語)は,人間の言語・概念に近づけて設計されてたプログラム言語である.\cite{高水準言語}\par
本実験の目的は,アセンブリ言語や機械語などの低級言語は{\ttfamily Java}などの高級言語と比べて,実装に関して大きく異なる点があるかどうかを明らかにすることであり,コードの複雑さやコード量がどうなるかを明らかにすることである.
\subsection{実験の方法}
実装に関して大きく異なる点を明らかにするため,低級言語と高級言語の一般的な実装手順を書き出し比較をする.低級言語はアセンブリ言語,高級言語は{\ttfamily Java}を利用する.\par
さらに,コードの複雑さやコード量がどうなるかを明らかにするために,実際に2つの言語で書いた同一のアルゴリズムに対して,行数や比較回数を比べる.
\subsection{実験の結果}