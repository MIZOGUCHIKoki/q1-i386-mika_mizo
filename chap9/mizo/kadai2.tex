\chapter{アセンブリ言語と高級言語の実装に関する違い}
\section{実験の目的}
先にも述べたように,アセンブリ言語は機械語を1対1に記した記法である.それに対して高級言語(高水準言語)は,人間の言語・概念に近づけて設計されてたプログラム言語である.\cite{高水準言語}\par
本実験の目的は,アセンブリ言語や機械語などの低級言語は{\ttfamily Java}などの高級言語と比べて,実装に関して大きく異なる点があるかどうかを明らかにすることであり,コードの複雑さやコード量がどうなるかを明らかにすることである.
\section{実験の方法}
実装に関して大きく異なる点を明らかにするため,低級言語と高級言語の一般的な実装手順を書き出し比較をする.低級言語はアセンブリ言語,高級言語は{\ttfamily Java}を利用する.\par
さらに,コードの複雑さやコード量がどうなるかを明らかにするため,実際に2つの言語で書いた同一のアルゴリズムに対して,行数や条件分岐,ループの回数を比べる.\par
{\ttfamily Java}におけるループ回数は{\ttfamily for}文の個数,比較回数は{\ttfamily if}文の個数とし,アセンブリ言語における比較回数は{\ttfamily cmp}の個数,ループ回数は一定条件下で上の行のラベルにジャンプする回数とする.評価値は
\[評価値=行数+比較回数+ループ回数\]と定義し,評価値とコードの複雑さは比例するものとする.
\section{実験の結果}
\ref{src:sort.s},\ref{src:testsort.s}は,アセンブリ言語で記述したプログラム,\ref{src:sort.java}は{\ttfamily Java}で記述したプログラムである.いずれも,入力データを受け取りバブルソートアルゴリズムで整列してその整列結果を1行ずつ出力するプログラムであり,入力と出力は一致している.それぞれの行数と比較回数を\ref{tbl:行数の比較}に示す.\par
\begin{table}[h]
    \centering
    \caption{行数とループ・比較回数}
    \label{tbl:行数の比較}
    \begin{tabular}{p{4cm}p{3cm}wc{2cm}wc{2cm}wc{2cm}wc{2cm}}
        \multicolumn{1}{c}{記述言語} & \multicolumn{1}{c}{ファイル名} & \multicolumn{1}{c}{行数} & \multicolumn{1}{c}{比較回数} & \multicolumn{1}{c}{ループ回数} & 評価値                 \\
        \hline
        \multirow{2}{*}{アセンブリ言語} & {\ttfamily sort.s}        & 44                     & 3                        & 2                         & \multirow{2}{*}{77} \\
                                 & {\testsort}               & 26                     & 1                        & 1                                               \\
        \hline
        {\ttfamily Java}         & {\ttfamily sort.java}     & 19                     & 1                        & 2                         & 22                  \\
        \hline
    \end{tabular}
\end{table}
\section{考察}
実験の結果より,両言語の評価値を比べるとアセンブリ言語の評価値の方が{\ttfamily Java}に比べて3.5倍であることが確認できる.1番目立った違いは行数だろう.アセンブリ言語で記述したものに比べて{\ttfamily Java}で記述したアルゴリズムは約\(1/3\)とより簡潔に記述できることが分かる.\par
その要因として{\ttfamily Java}のループに使われる{\ttfamily for}文は,比較とループ,ループ変数の定義と処理を1行で行うことのできることが挙げられる.\par
今回の実験で,低級言語であるアセンブリ言語の方が,高級言語である{\ttfamily Java}よりも複雑でゴードの量も多くなることが分かった.ただし,この実験では独自の指標でコードの複雑さを測っているため,一般的な複雑の指標である,\textbf{サイクロマティック複雑度(循環的複雑度)}で計測できていない.