\chapter{アセンブリ言語による整列アルゴリズム記述可否の検証}\label{chap1}
\section{実験の目的}
高級プログラミング言語,{\ttfamily Java, C, Python}などは,『コンパイラ』と呼ばれる装置を通して機械語に書き換えられ,コンピュータで実行されている.\par
それに対して,アセンブリ言語は各機械語命令につけられた「意味する名前」(ニーモニック;mnemonic)を使ってプログラムを表記する表記法である.\cite[第1章]{pl2text}
また,アセンブリ言語表記を機械語のビット列に変換する作業をアセンブルと言い,それを行うソフトウェアをアセンブラと言う.
つまり,コンパイラとアセンブラは別物であり,アセンブラは機械語の表記を変えたものである故にコンピュータへの命令を1対1で書き換えるものである点がコンパイラと大きく違う点である.\par
本実験課題の目的は,このようなアセンブリ言語・機械語に対して,コンパイラを使わずに整列アルゴリズムを直接技術することが可能であるか確認することである.
\section{プログラムの仕様}
\(0\)以上\(2^{31}\)未満の自然数列に対して昇順に整列するアルゴリズムをアセンブリ言語で記述する.その際,\testsort ファイルが\sort サブルーチンを呼び出して整列を行う.
検証したい自然数列は,\testsort 内の{\ttfamily data1}にダブルワードで,データの個数は{\ttfamily ndata1}で定義しており,\testsort 実行時{\ttfamily data1}に格納してある自然数列が\print サブルーチンによって出力される.\par
{\ttfamily data1}の先頭番地は{\ttfamily EBX},{\ttfamily ndata1}は{\ttfamily ECX}に格納する.\sort の呼び出し前後で他の汎用レジスタの値は変化しないように設計されている.内部処理の概要を\ref{kadai1:abs}に示す.整列アルゴリズムは,\textbf{選択ソートアルゴリズム}\footnote{入力データの最大値を見つけ,それを整列アルゴリズムから除外することを繰り返し行うアルゴリズム\cite[p.49]{アルゴリズムとデータ構造}}を元に作成している.
\begin{figure}[H]
    \centering
    \caption{処理概要}
    \label{kadai1:abs}
    \tikzset{mynode/.style={rectangle,rounded corners,draw,minimum height=1cm,minimum width=3cm,text centered}}
    \begin{tikzpicture}
        \node at (0,0)(call){{\ttfamily call sort}};
        \node[below=0.5cm of call](print){{\ttfamily call print\_eax}};
        \node[right=2cm of call,text width=5cm](sort){{\ttfamily sort}サブルーチン};
        \node[left=2cm of print,text width=5cm](printeax){\print サブルーチン};
        \draw[-Stealth,very thick](call.east)to(sort.west);
        \draw[-latex,very thick](call)to(print);
        \draw[-Stealth,very thick](print)to[bend left=30](printeax);
        \draw[-Stealth,very thick](printeax)to[bend left=30](print);
        \node at ($(print)!0.5!(printeax)+(0,-1.5cm)$){\sort されたデータを1行ずつ表示};
        \node[inner sep=0.2cm,fit={(print)(call)},draw,dotted](){};
    \end{tikzpicture}
\end{figure}
\section{実験}
\subsection{実験方法}
\ref{usingPC}に示すコンピュータとソフトウェアを利用して,アセンブリ言語で書いた整列プログラムを実行可能ファイルに変換し実行した.
実行結果を確認し,整列されているか確認するとともに,{\ttfamily Java}で記述したプログラムとアセンブリプログラムを比較してどの部分をどのようにアセンブリ言語化したかを確認する.
\begin{lstlisting}[frame={single},numbers={none},breakindent={0pt},language={Bash},caption={使用したコンピュータとソフトウェア},label={usingPC}]
$ uname -a
Linux KUT20VLIN-462 5.4.0-70-generic #78~18.04.1-Ubuntu SMP Sat Mar 20 14:10:07 UTC 2021 x86_64 x86_64 x86_64 GNU/Linux
$ nasm --version
NASM version 2.13.02
$ ld --version
GNU ld (GNU Binutils for Ubuntu) 2.30
$ java --version
openjdk 11.0.10 2021-01-19
\end{lstlisting}
\print{\ttfamily .s},\testsort,\sort{\ttfamily .s}をそれぞれアセンブルして実行する.
\begin{lstlisting}[frame={single},numbers={none},breakindent={0pt},language={Bash},caption={実行したコマンド},label={command1}]
$ nasm -felf print_eax.s ; nasm -felf test_sort.s ; nasm -felf sort.s
$ ld -m elf_i386 print_eax.o test_sort.o sort.o
$ ./a.out
\end{lstlisting}
\subsection{実験結果}
実験結果と期待される結果を\ref{tbl:execute}に示す.自然数列を昇順に整列されていることが確認できる.\par
\begin{table}[H]
    \centering
    \caption{実験結果と比較}
    \label{tbl:execute}
    \begin{tabular}{l|l}
        入力   & {\ttfamily 1 3 5 7 9 2 4 6 8 0 1 2} \\
        \hline
        出力   & {\ttfamily 0 1 1 2 2 3 4 5 6 7 8 9} \\
        期待出力 & {\ttfamily 0 1 1 2 2 3 4 5 6 7 8 9}
    \end{tabular}
\end{table}
{\ttfamily Java}で記述した選択ソート(\ref{src:java_buble})と,アセンブリ言語で記述した選択ソート(\ref{src:assembuly_buble})を比較して,言語化の箇所を確認する.前者は入力データ列を{\ttfamily data}とし,後者は{\ttfamily EBX}に入力データ列の先頭番地,{\ttfamily ECX}入力データ列の個数を格納している.ソースコードの各処理の内容と,2つの言語による対応を\ref{tbl:ソースコードの行対応}に記す.
\begin{table}[htbp]
    \centering
    \caption{ソースコードの行対応}
    \label{tbl:ソースコードの行対応}
    \begin{tabular}{p{10cm}p{2cm}p{2cm}}
        \multicolumn{1}{c}{処理内容}                              & \multicolumn{1}{c}{\ref{src:java_buble}} & \multicolumn{1}{c}{\ref{src:assembuly_buble}} \\
        \hline
        ループ処理1の条件比較・変数処理                                      & 3                                        & 2 - 3, 25, 26                                 \\
        最大値の更新                                                & 4                                        & 5                                             \\
        最大値インデックスの更新                                          & 5                                        & 6                                             \\
        ループ処理2の条件比較・変数処理                                      & 6                                        & 7 - 9, 18, 19                                 \\
        最大値の比較・更新                                             & 7 - 11                                   & 10 - 17                                       \\
        {\ttfamily data[max\_index]}と{\ttfamily data[i]}の入れ替え & 12 - 14                                  & 21 - 24                                       \\
        \hline
    \end{tabular}
\end{table}\\
\begin{minipage}[t]{0.5\linewidth}
    \centering
    \begin{lstlisting}[caption={{\ttfamily Java}}, label={src:java_buble}, language={Java},frame={left}]
int max_index = 0;
int max = 0;
for (int i = data.length - 1; i > 0; i--) {
  max = data[0];
  max_index = 0;
  for (int j = 1; j <= i; j++) {
      if (data[j] >= max) {
              max = data[j];
              max_index = j;
          }
      }
      int m = data[max_index];
      data[max_index] = data[i];
      data[i] = m;
}
    \end{lstlisting}
    \begin{flushleft}
        \begin{framed}
            \noindent\textbf{注)}\par
            \ref{src:java_buble},\ref{src:assembuly_buble}いずれも,整列アルゴリズムの部分のみ掲載している.
        \end{framed}
    \end{flushleft}
\end{minipage}
\hspace{3em}
\begin{minipage}[t]{0.4\linewidth}
    \begin{lstlisting}[frame={left},caption={アセンブリ},label={src:assembuly_buble}]
loop0:
  cmp   ecx,  0
  jle   endp
  mov   edx,  [ebx]
  mov   eax,  0    
  mov   edi,  1
  loop1:
    cmp   edi,  ecx
    jg    loop0l
    mov   esi,  [ebx + edi*4]
    cmp   esi,  edx
    jge   then
    jmp   endif
    then:
      mov edx,  [ebx + edi*4]
      mov eax,  edi          
    endif:
      inc edi
      jmp loop1
  loop0l:
    mov   esi,  [ebx + eax*4]
    mov   edi,  [ebx + ecx*4]
    mov   [ebx + eax*4],  edi
    mov   [ebx + ecx*4],  esi
    dec   ecx
    jmp   loop0
\end{lstlisting}
\end{minipage}
\section{考察}
実験結果より自然数列を整列アルゴリズムをアセンブリ言語で記述できることがわかった.\par
ただ,これはあくまで\(0\)以上\(2^{31}\)未満の自然数に限った整列アルゴリズムであるため,負の整数やその他の有理数などを対象にした整列アルゴリズムが記述可能であるかは,この実験では検証できていない.